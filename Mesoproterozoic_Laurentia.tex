\documentclass[11pt,letterpaper]{article}

\usepackage[left]{lineno}
\usepackage{natbib}
\usepackage{setspace}
\usepackage{graphicx}
\doublespacing

\raggedright
\textwidth = 6.5 in
\textheight = 9 in
\oddsidemargin = 0.0 in
\evensidemargin = 0.0 in
\topmargin = 0.0 in
\headheight = 0.0 in
\headsep = 0.0 in
\parskip = 0.1 in
\parindent = 0.2 in

\usepackage[aboveskip=1pt,labelfont=bf,labelsep=period,justification=raggedright,singlelinecheck=off]{caption}
%commands that were used to generate a PDF that was copied to make the .docx version for GSAB
%\pagestyle{empty}
%\usepackage[nomarkers,figuresonly]{endfloat}

\begin{document}

\begin{flushleft}
{\LARGE \textbf{The Grenvillian orogeny as a major turning point in the tectonic evolution of Laurentia}}

\noindent\textit{This article is in preparation for the GSA Memoir Tectonic Evolution of Laurentia and North America}
\end{flushleft}

%\linenumbers

\section*{Abstract}
Following the initial assembly of Archean provinces to form Laurentia ca. 1800 Ma, there was more the 600 million years of accretionary growth along southeast Laurentia. The Grenvillian orogeny, which initiated ca. 1090 Ma in the late Mesoproterozoic, was a major turning point in Laurentia's tectonic evolution. It marked the end of this extended interval of subduction and accretion, terminated and inverted the major Midcontinent Rift whose continued development could have split Laurentia down the middle, and resulted in the formation of the supercontinent Rodinia.

\subsection*{Shawinigan (+ Llano) orogenesis}
Detail the record of accretionary orogenesis which marks the last pulse of the prolonged latest Paleoproterozoic through Mesoproterozoic subduction-accretion beneath/to SE Laurentia.

\subsection*{Ca. 1170 to 1140 Ma magmatic activity}
Describe the AMCG suite along the SE Laurentian margin during the latter stages of the Shawinigan orogeny (ca. 1170-1140 Ma) as well as the ca. 1140 Ma Midcontinent magmatic activity (Abitibi dikes + coeval lamprophyre dikes + Corson diabase).

\subsection*{The development and failure of the Midcontinent Rift}
Detail the timing of Midcontinent Rift development and magmatic activity drawing connections with the SW Laurentia LIP.

\subsection*{Grenvillian orogenesis and coeval basin development}
Summarize the history of Grenvillian orogenesis including a review of its timing and dynamics from initiation through to extensional collapse. Evaluate connections with the inversion of the Midcontinent Rift which can be used to critically evaluate hypotheses for Midcontinent Rift failure. Summarize the evidence for the position of the Grenville Front from where it is exposed in Canada down through the US based on both geologic and geophysical data.

\subsection*{Laurentia's role in Rodinia and its assembly}
Present the state-of-the-art in terms of paleogeographic constraints. Highlight the rapid motion of Laurentia from high-latitudes to the equator leading up to Grenvillian orogenesis. Critically evaluate proposed conjugate cratons in terms of available paleomagnetic data and geochemical data (e.g. Pb isotope arrays). 

\end{document}
%of Laurentian record with those from other continental fragments in 
%Rodinia

%A few thoughts:
%
%1. Compiling dates of Grenvillian Orogeny and Shawinigan Orogeny:
%It's always good to have compilations, for sure, but whether this would 
%lead to clarity about timing of GO and SO seems less certain to me. Age 
%determinations in high-grade metamorphic rocks are commonly complicated 
%by the presence of igneous cores and metamorphic rims, and many of the 
%existing age data for the GP in Canada were done 20-40 years ago by the 
%traditional TIMS method of Tom Krogh --> whole-grain Zrn dissolution 
%(both single grain and multi-grain samples) after physical abrasion of 
%grain rims - mostly without prior CL or BSE imaging. Krogh and followers 
%generally separated populations into prismatic and rounded soccer ball 
%grains, interpreted as igneous and metamorphic respectively, but the 
%lack of imaging raises a flag for me. Moreover, although the results are 
%commonly precise, their geological significance is not always easy to 
%interpret. For instance, Ottawan ages of 1090, 1060 and 1020 Ma have 
%been determined in different samples/ parts of the orogen, but their 
%significance can only be inferred from other commonly independent data. 
%Moreover, the geological significance of a date for a regional 
%metamorphic event with an uncertainty of +/- 2 Ma is questionable. It 
%must be a specific reaction that is being dated, not the long-lived 
%metamorphic event involving the cycle of crustal burial, heating, 
%exhumation and cooling.
%
%In summary, I think there is a need to revisit many of these samples / 
%results and run them with a method that links the geochronometer to the 
%reaction history of the sample; SHRIMP dating of Zrn rims and cores, or 
%Mnz dating coupled with REE zoning, as practised at Mike's lab, look 
%like the currently best ways to go. Then we will be in a better position 
%to compare ages along the length of the orogen.
%
%2. WRT the time of initial collision between Laurentia and Amazonia, all 
%ages in the GP are metamorphic ages, so this info is not directly 
%available from Laurentia as far as we know. However, there is a hiatus 
%in metamorphic ages between what has been inferred to be the end of the 
%Shawinigan Orogeny at ca. 1140 and the onset of the GO at ca. 1090 Ma 
%which presumably brackets the time of collision.
%
%3. I agree that it's time to re-assess the link / relationship between 
%the MCR and the GO. As mentioned in one of the presentations, there is 
%evidence that magmatism in the MCR started as early as ~1060 Ma 
%(kimberlite magmatism; paper by Heaman et al - in 2000s I think), which 
%coincides with AMCG magmatism, emplacement of Abitibi dyke etc., and the 
%Shawinigan orogeny. As far as I know, this temporal link has not been 
%explored in detail. As discussed in the meeting, the volumetric peak of 
%magmatism appears to have coincided with the early Ottawan phase of the 
%GO; this temporal link is also worth additional scrutiny... Of course, 
%temporal links as just that, and do not prove common causality.
%That said, I have always found the analysis of Bill Cannon regarding the 
%closure of the MCR during the Ottawan phase to be quite compelling.
%
%4. An elephant in the room is the tectonic setting of emplacement of the 
%AMCG suite along the SE Laurentian margin during the latter stages of 
%the Shawinigan orogeny (ca. 1170-1140 Ma). This enormously volumetric 
%suite must have been a result of an important magmatic process in the 
%mantle. All proposed explanations so far seem to have come up short, and 
%the problem has only become more intractable since the discovery that 
%the megacrysts in some AMCG suites have ages several hundred M.y. older 
%than their host magmas (paper by Bybee et al a few years ago).
%In addition, given that the evidence for the Shawinigan Orogeny is 
%widespread in the western GP and the inliers in the Appalachians, but 
%much less evident in the eastern GP, the temporal overlap between the SO 
%and AMCG suite magmatism may be coincidental rather than process-driven.
%
%5. Turning points: Not quite sure how to define a turning point, but 
%here are a few suggestions to consider: Turning point (i) closure of a 
%large Pacific-type ocean following the prolonged latest Paleoproterozoic 
%and Mesoproterozoic subduction-accretion beneath/to SE Laurentia, 
%leading to formation of the accretionary GPAO; (ii) Formation of double 
%crust and an orogenic plateau in the Ottawan phase of the Grenvillian 
%Orogeny; (iii) Collapse of the Ottawan double crust in the GO; (iv) 
%Initiation of subduction of the Laurentian plate (double bivergent 
%subduction) during the Rigolet phase; (v) Rodinia assembly - comparison 
%of Laurentian record with those from other continental fragments in 
%Rodinia; (vi) break-up of Rodinia, and opening of new ocean basins.
%
%I realize reading this through that it is too Grenvillian-centred. Need 
%to also include Mesoproterozoic and Neoproterozoic sedimentation in 
%basins on and surrounding Laurentia.
%
%Enough for now; hope you find these thoughts useful...
%
%Best
%
%Toby
%>> * provide a framework to discuss the Midcontinent Rift developing
%>> during the orogenic lull between Shawinigan and the Ottawan
%>> orogenesis
%>> 
%>> * enable an assessment of whether the hypothesis that cessation of
%>> the rift is associated with the onset of Ottawan orogenesis is
%>> consistent with the best available constraints.
%>> * provide a resource for those seeking to attribute detrital
%>> zircon data to Laurentia tectonomagmatic events
%>> * enable discussion of the timing of collisional orogenesis as